\RequirePackage{currfile}
\documentclass[11pt]{article}
\usepackage[utf8]{inputenc}
\usepackage[ngerman]{babel}
\usepackage{libertine}
\usepackage[a4paper]{geometry}
\usepackage{parskip}
\usepackage{amsmath, amsthm, amssymb} 
\usepackage{mathtools}
\usepackage{booktabs}
\usepackage{tabularx}
\usepackage{enumitem}
\usepackage{graphicx}
\usepackage{xcolor}
\usepackage{float}
\usepackage{wrapfig}
\usepackage[makeroom]{cancel}
\usepackage{multicol}
\usepackage{multirow}
\usepackage{vwcol} 	 	% Provides variable multicol
\usepackage{commath} 	% Provides good differentials
\usepackage{esint} 		% Provides various fancy integral symbols
\usepackage{siunitx} 	% Provides good units
\usepackage{nicefrac}
\usepackage{dashrule}
\usepackage{minted}

\usepackage{csquotes}
\MakeOuterQuote{"}

% http://packages.oth-regensburg.de/ctan/macros/latex/contrib/currfile/currfile.pdf
% % https://tex.stackexchange.com/a/54891/116525

\makeatletter
\newcommand{\toleftmargin}[1]{\par\noindent\hspace{-\@totalleftmargin}\parbox[t]{\textwidth}{#1}}
% \newenvironment{javaenv}{\begin{minted}[linenos,firstnumber=last,autogobble,xleftmargin=-\@totalleftmargin]{java}}{}
\newlength{\leftmargins}
\makeatother
% https://tex.stackexchange.com/a/481735

\def\nrub{6}
\newcommand{\blanko}[0]{\textcolor{white}{.}}

\usepackage{hyperref}
\hypersetup{
	pdftitle={Betriebsysteme (WS19/20) Übungsblatt \nrub ~- 12141043},
	pdfauthor={Yudong SUn},
	bookmarksnumbered=true,
	bookmarksopen=true,
	bookmarksopenlevel=2,
	pdfstartview=Fit,
	pdfpagemode=UseOutlines,
	colorlinks=true,
	linkcolor=black,
	filecolor=magenta,      
	urlcolor=blue
}
\urlstyle{same}

\renewcommand{\ttdefault}{cmtt}

\usepackage{fancyhdr}
 
\pagestyle{fancy}
\fancyhf{}
\fancyhead[RO]{Yudong Sun / \texttt{12141043}}
\fancyhead[LO]{Übungsblatt \nrub}
\fancyhead[LE]{\texttt{12141043} / Yudong Sun}
\fancyhead[RE]{Übungsblatt \nrub}
\cfoot{\thepage}

\title{Betriebsysteme (WS19/20)\\Übungsblatt \nrub}
\author{Yudong Sun\\\texttt{12141043}}
\date{\today}

\begin{document}

\maketitle

% % Besprechung 2019-12-12
\begin{enumerate}[label={T\arabic*},start=38]
	\item Kernel Level Threads vs User Level Threads
		\begin{enumerate}
			\item KLT (Kernel = Split threads before passing to user)
			\item Die Prozessor ist die Marke. 2 Marke in S1 heißt, dass P1 2 Prozessor benutzt. 
			\item Es hande sich beim Erreichbarkaisgraphen um einen zylischen, stark zusammenhngenden Graphen. D.h. jeder Markierung aus ist jede andere durch Anwenden der Transitionen erreichbar.
			\item Echte Teilgraph
		\end{enumerate}
	\item (c) Ein Petri-Netz heißt fair, wenn in jedem unendlichen Durchlauf alle Transitionen endlich oft geschaltet werden.
\end{enumerate}

Julian Hager
Gruppe 08
\begin{enumerate}[label={Aufgabe H\arabic*},start=55]
    \item
        \begin{enumerate}
            \item Man braucht $\frac{256}{8} = 32$ eindeutige Addresse, das heißt 5 bits.
            \item $P_1$ geladen:
                \vspace{-\parskip}
                \begin{figure}[h!]
                    \centering
                    \begin{tikzpicture}[level/.style={sibling distance=50mm/#1},level distance=10mm]
                        \tikzstyle{free}=[square,thick,draw=gray!75,fill=gray!20,minimum size=6mm]
                        \tikzstyle{splitused}=[circle,thick,draw=gray!75,fill=gray!20,minimum size=6mm]

                        \begin{scope}
                            \node [splitused,draw] (z) {}
                                child
                                {
                                    node [splitused,draw] (zz) {}
                                    child 
                                    {
                                        node [splitused,draw] (zzz) {}
                                        child 
                                        {
                                            node [splitused,draw] (zzzz) {}
                                            child { node [splitused,draw,label=left:{\small \begin{tabular}{c}\SI{16}{\mebi\byte}\\\texttt{00000} \end{tabular}}] (zzzzz) {$P_1$} }
                                            child { node [free,draw,label=right:{\small \SI{16}{\mebi\byte}}] (zzzzo) {} }
                                        }
                                        child { node [free,draw,label=right:{\small \SI{32}{\mebi\byte}}] (zzzo) {} }
                                    }
                                    child { node [free,draw,label=right:{\small \SI{64}{\mebi\byte}}] (zzo) {} }
                                }
                                child { node [free,draw,label=right:{\small \SI{128}{\mebi\byte}}] (zo) {} }
                            ;
                        \end{scope}
                    \end{tikzpicture}
                \end{figure}

                $P_2$ geladen:
                \vspace{-\parskip}
                \begin{figure}[h!]
                    \centering
                    \begin{tikzpicture}[level/.style={sibling distance=50mm/#1},level distance=10mm]]
                        \tikzstyle{free}=[square,thick,draw=gray!75,fill=gray!20,minimum size=6mm]
                        \tikzstyle{splitused}=[circle,thick,draw=gray!75,fill=gray!20,minimum size=6mm]

                        \begin{scope}
                            \node [splitused,draw] (z) {}
                                child
                                {
                                    node [splitused,draw] (zz) {}
                                    child 
                                    {
                                        node [splitused,draw] (zzz) {}
                                        child 
                                        {
                                            node [splitused,draw] (zzzz) {}
                                            child { node [splitused,draw,label=left:{\small \begin{tabular}{c}\SI{16}{\mebi\byte}\\\texttt{00000} \end{tabular}}] (zzzzz) {$P_1$} }
                                            child { node [free,draw,label=right:{\small \SI{16}{\mebi\byte}}] (zzzzo) {} }
                                        }
                                        child { node [free,draw,label=right:{\small \SI{32}{\mebi\byte}}] (zzzo) {} }
                                    }
                                    child { node [splitused,draw,label=right:{\small \begin{tabular}{c}\SI{64}{\mebi\byte}\\\texttt{01000} \end{tabular}}] (zzo) {$P_2$} }
                                }
                                child { node [free,draw,label=right:{\small \SI{128}{\mebi\byte}}] (zo) {} }
                            ;
                        \end{scope}
                    \end{tikzpicture}
                \end{figure}

                \pagebreak
                $P_3$ geladen:
                \vspace{-\parskip}
                \begin{figure}[h!]
                    \centering
                    \begin{tikzpicture}[level/.style={sibling distance=50mm/#1},level distance=10mm]]
                        \tikzstyle{free}=[square,thick,draw=gray!75,fill=gray!20,minimum size=6mm]
                        \tikzstyle{splitused}=[circle,thick,draw=gray!75,fill=gray!20,minimum size=6mm]

                        \begin{scope}
                            \node [splitused,draw] (z) {}
                                child
                                {
                                    node [splitused,draw] (zz) {}
                                    child 
                                    {
                                        node [splitused,draw] (zzz) {}
                                        child 
                                        {
                                            node [splitused,draw] (zzzz) {}
                                            child { node [splitused,draw,label=left:{\small \begin{tabular}{c}\SI{16}{\mebi\byte}\\\texttt{00000} \end{tabular}}] (zzzzz) {$P_1$} }
                                            child { node [free,draw,label=right:{\small \SI{16}{\mebi\byte}}] (zzzzo) {} }
                                        }
                                        child { node [free,draw,label=right:{\small \SI{32}{\mebi\byte}}] (zzzo) {} }
                                    }
                                    child { node [splitused,draw,label={\small \begin{tabular}{c}\SI{64}{\mebi\byte}\\\texttt{01000} \end{tabular}}] (zzo) {$P_2$} }
                                }
                                child 
                                {
                                    node [splitused,draw] (zo) {}
                                    child { node [splitused,draw,label=below:{\small \begin{tabular}{c}\SI{64}{\mebi\byte}\\\texttt{10000} \end{tabular}}] (zoz) {$P_3$} }
                                    child { node [free,draw,label=right:{\small \SI{64}{\mebi\byte}}] (zoo) {}}
                                }
                            ;
                        \end{scope}
                    \end{tikzpicture}
                \end{figure}

                $P_4$ geladen:
                \vspace{-\parskip}
                \begin{figure}[h!]
                    \centering
                    \begin{tikzpicture}[level/.style={sibling distance=50mm/#1},level distance=10mm]]
                        \tikzstyle{free}=[square,thick,draw=gray!75,fill=gray!20,minimum size=6mm]
                        \tikzstyle{splitused}=[circle,thick,draw=gray!75,fill=gray!20,minimum size=6mm]

                        \begin{scope}
                            \node [splitused,draw] (z) {}
                                child
                                {
                                    node [splitused,draw] (zz) {}
                                    child 
                                    {
                                        node [splitused,draw] (zzz) {}
                                        child 
                                        {
                                            node [splitused,draw] (zzzz) {}
                                            child { node [splitused,draw,label=left:{\small \begin{tabular}{c}\SI{16}{\mebi\byte}\\\texttt{00000} \end{tabular}}] (zzzzz) {$P_1$} }
                                            child { node [splitused,draw,label=right:{\small \begin{tabular}{c}\SI{16}{\mebi\byte}\\\texttt{00010} \end{tabular}}] (zzzzo) {$P_4$} }
                                        }
                                        child { node [free,draw,label=right:{\small \SI{32}{\mebi\byte}}] (zzzo) {} }
                                    }
                                    child { node [splitused,draw,label={\small \begin{tabular}{c}\SI{64}{\mebi\byte}\\\texttt{01000} \end{tabular}}] (zzo) {$P_2$} }
                                }
                                child 
                                {
                                    node [splitused,draw] (zo) {}
                                    child { node [splitused,draw,label=below:{\small \begin{tabular}{c}\SI{64}{\mebi\byte}\\\texttt{10000} \end{tabular}}] (zoz) {$P_3$} }
                                    child { node [free,draw,label=right:{\small \SI{64}{\mebi\byte}}] (zoo) {}}
                                }
                            ;
                        \end{scope}
                    \end{tikzpicture}
                \end{figure}
            \item Die freie Speicherplätze sind in dem Speicher geteilt. Da die freie Speicherplätze nicht in einem Block sind, ist es nicht gemeinsam nutzbar. Dabei entsteht durch das Buddy-System externe und interne Fragmentierung. Insgesamt steht nur noch \SI{96}{\mebi\byte} getrennt in \SI{32}{\mebi\byte} und \SI{64}{\mebi\byte} Segmente für weitere Programme zur Verfügung. 

            \item Nein, $P_5$ kann nicht in den Speicher geladen werden. Es gibt in dem oberen Buddy-Baum nur 2 freie Segmente, jeweils \SI{32}{\mebi\byte} und \SI{64}{\mebi\byte} groß. Da \SI{96}{\mebi\byte} größer als \SI{64}{\mebi\byte} ist, steht keine freie Segmente für $P_5$ zur Verfügung. 

            \item $P_4$ terminiert:
                \vspace{-\parskip}
                \begin{figure}[h!]
                    \centering
                    \begin{tikzpicture}[level/.style={sibling distance=50mm/#1},level distance=10mm]]
                        \tikzstyle{free}=[square,thick,draw=gray!75,fill=gray!20,minimum size=6mm]
                        \tikzstyle{splitused}=[circle,thick,draw=gray!75,fill=gray!20,minimum size=6mm]

                        \begin{scope}
                            \node [splitused,draw] (z) {}
                                child
                                {
                                    node [splitused,draw] (zz) {}
                                    child 
                                    {
                                        node [splitused,draw] (zzz) {}
                                        child 
                                        {
                                            node [splitused,draw] (zzzz) {}
                                            child { node [splitused,draw,label=left:{\small \begin{tabular}{c}\SI{16}{\mebi\byte}\\\texttt{00000} \end{tabular}}] (zzzzz) {$P_1$} }
                                            child { node [free,draw,label=right:{\small \SI{16}{\mebi\byte}}] (zzzzo) {} }
                                        }
                                        child { node [free,draw,label=right:{\small \SI{32}{\mebi\byte}}] (zzzo) {} }
                                    }
                                    child { node [splitused,draw,label={\small \begin{tabular}{c}\SI{64}{\mebi\byte}\\\texttt{01000} \end{tabular}}] (zzo) {$P_2$} }
                                }
                                child 
                                {
                                    node [splitused,draw] (zo) {}
                                    child { node [splitused,draw,label=below:{\small \begin{tabular}{c}\SI{64}{\mebi\byte}\\\texttt{10000} \end{tabular}}] (zoz) {$P_3$} }
                                    child { node [free,draw,label=right:{\small \SI{64}{\mebi\byte}}] (zoo) {}}
                                }
                            ;
                        \end{scope}
                    \end{tikzpicture}
                \end{figure}

                $P_1$ terminiert:
                \vspace{-\parskip}
                \begin{figure}[h!]
                    \centering
                    \begin{tikzpicture}[level/.style={sibling distance=50mm/#1},level distance=10mm]]
                        \tikzstyle{free}=[square,thick,draw=gray!75,fill=gray!20,minimum size=6mm]
                        \tikzstyle{splitused}=[circle,thick,draw=gray!75,fill=gray!20,minimum size=6mm]

                        \begin{scope}
                            \node [splitused,draw] (z) {}
                                child
                                {
                                    node [splitused,draw] (zz) {}
                                    child { node [free,draw,label=right:{\small \SI{64}{\mebi\byte}}] (zzz) {} }
                                    child { node [splitused,draw,label=below:{\small \begin{tabular}{c}\SI{64}{\mebi\byte}\\\texttt{01000} \end{tabular}}] (zzo) {$P_2$} }
                                }
                                child 
                                {
                                    node [splitused,draw] (zo) {}
                                    child { node [splitused,draw,label=below:{\small \begin{tabular}{c}\SI{64}{\mebi\byte}\\\texttt{10000} \end{tabular}}] (zoz) {$P_3$} }
                                    child { node [free,draw,label=right:{\small \SI{64}{\mebi\byte}}] (zoo) {}}
                                }
                            ;
                        \end{scope}
                    \end{tikzpicture}
                \end{figure}
        \end{enumerate}
    \item
        \begin{enumerate}
            \item Eine Seite enthält $16 \times 1024 = 2^{4} \times 2^{10} = 2^{14}$ individuelle adressierbare Einheiten. Dabei benötigen wir 14 Bits zur Adressierung eines Wortes innerhalb einer Seite. Man bezeichnet diesen Teil der Adresse als die Relativadresse bzw. Offset. 

            \item Der verfügbare Arbeitsspeicher umfasst $\SI{64}{\mebi\byte} = 64 \times 2^{10}~\SI{}{\kibi\byte}$ mit jeder Seite \SI{16}{\kibi\byte}. Daher gibt es
            \begin{equation*}
                \frac{64 \times 2^{10}}{16} = \frac{2^6 \times 2^{10}}{2^4} = 2^{12}
            \end{equation*}
            individuelle adressierbare Seiten. Wir benötigen dabei 12 Bits zur Adressierung einer physischen Seite.

            \item Ein Adressbus hat eine Breite von 28 Bits. Für jede Seite braucht man 14 Bits zur Adressierung eines Wortes. Es steht übrig $28 - 14 = 14$ Bits zur Verfügung für die Adressierung einer virtuellen Seite. Damit kann man maximal $2^{14} = 16384$ virtuelle Seiten adressieren. 

            \item Um den maximal adressierbaren virtuellen Speicher zur Verfügung stellen zu können, brauchen wir einen Hintergrundspeicher mit einer Größe $2^{14} /~2^{12} = 2^2 = 4$ mal so groß wie den physischen Speicher. Das heißt, der Hintergrundspeicher muss mindesten $4 \times \SI{64}{\mebi\byte} = \SI{256}{\mebi\byte}$ groß sein. 

            \item Man beobachtet interne Fragmentierung bei Anwendung von Paging. Paging ist ein Form der statischen bzw. festen Partitionierung. Alle bis auf der letzte Seite sind guarantiert voll belegt, nur die letzte Seite kann zum Teil leer sein. Deshalb gibt es bei Paging nur interne Fragmentierung und keine externe Fragmentierung.
        \end{enumerate}
    \item Sehen Sie bitte \texttt{u11-h57.txt}
\end{enumerate}

\end{document}
