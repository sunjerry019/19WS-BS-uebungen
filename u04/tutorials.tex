% Besprechung 2019-11-14
\begin{enumerate}[label={T\arabic*},start=18]
	\item 
		\begin{enumerate}
			\item
				== Zustand ==
				New:	Der Patient befindet sich im Eingangsbereich
				Rdsp:	Patient im Wartezimmer. Könnte behandelt werden, sofern ein Behandlungszimmer frei ist. 
				Bksp:	Patient im Wartezimmer. Könnte nicht behandelt werden, weil er auf ein Ereignis (Röntgenbilder) wartet.
				Ready:	Patient im Behandlungszimmer, könnte behandelt werden.
				Blked:	wie Bksp., aber im Behandlungszimmer
				Runng:	Der Artz behandelt der Patienten
				Exit:	Der verlässt die Praxis.

				== Übergang ==
				New   -> Ready: Die Aufnahme ist erst abgeschlossen.
				RdSp  -> Ready: Die Patient wird in ein Behandlungszimmer gebeten
				Bksp  -> Ready: Die Patient wird in ein Behandlungszimmer gebeten
				Ready -> RdSp : Patient wird gebeten nochmals, im Wartezimmer Platz zu nehmen. 
				Blked -> BkSp : Patient wird gebeten nochmals, im Wartezimmer Platz zu nehmen. 
				Blked -> Ready: Patient sitzt Behandlungszimmer und die Röntgenbilder treffen ein.
				BkSp  -> RdSp : Patient sitzt Wartezimmer und die Röntgenbilder treffen ein.
				Ready -> Runng: Die Arzt beginnt die Behandlung (in Behandlungszimmer)
				Runng -> Ready: Arzt unterbrecht die Behdnlung (könnte weitergehen)
				Runng -> Blked: Arzt unterbrecht die Behdnlung, warten auf Röntgenbilder (im Behandlungszimmer)
				Runng -> Exit : Behandlung fertig ist. Patient verlässt die Praxis.

				READY BLOCKED -> BEHANDLUNGSZIMMER
				SUSPEND -> WARTEZIMMER

			\item Behandlungen bestehen meist aus mehreren Teiljobs zwischen diesen können (lange) Wartezeit entstehen z.B. Röntgenbilder
				-> Sinnvoll andere Patienten zu behandlen.
				Aber keine Unterbrechungen außerhalb dieser Teiljobs.

			\item 
				Nachteil (SJF): Patienten mit komplizierten Behandung müssen wahrscheinlich den ganzen Tag in der Praxis verbringen
				RR: Die Behandlungen dauern länger, aber die Wartezeiten wirken kürzer. 
				PS (Priority Scheduling): Notfälle schnell zu bearbeiten; Privatpatienten kann der Artz bevorzugen.
				MLFQ: Jeder Patient kommt anfangs schnell dran. Kurze routinen können also schnell abgearbeitet werden. Die Behandlung komplizierteer Fälle zögert sich jedoch hinaus.  
					? Is it possible for the 2nd queue to never reach?

			\item 
				- Die zufällife Auswahl per Zufall wird als unfair wahrgenomme, da es zu relativ wenigen Iterationen kommt. 
				- Beim Präemptiven-Scheduling ist die Anzahl der Unterbrechungen vergleichsweise hoch. Die Ausführshaufigkeit nähert sich also mit hoher Wahrscheinlichkeit dem Erwartungswert und wird fairer.

		\end{enumerate}
	\item 
		SRPT -> Every time section, recheck
		RR -> New processes always go to the end of the queue.
\end{enumerate}

Julian Hager
Gruppe 08