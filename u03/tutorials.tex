2019-11-07

\begin{enumerate}[label={T\arabic*},start=12]
	\item 
		\begin{enumerate}
			\item
				(i) nicht erlaubt
				(ii) Warten auf ereignis
				(iii) Nicht logisch
			\item 
				Scheduler != Dispatcher
					Scheduler:  Entscheidet wer wann dran ist
					Dispatcher:	Dispatching eines Ready-Prozesses

				New -> Ready: 			Anwendung gestartet
				Ready -> Running: 		Dispatched
				Timeout:				Laufenden Prozess Unterbrochen wird.
				Warte auf Ereignis: 	Warte auf benutzer Angabe
				Ergebniss tritt ein:	Eingabe ist geschehen
				Exit:					Von benutzer beendet wird
			\item Das Modell ändert sich nicht. Nicht Sinnvoll. Alles geht um 1 Prozess
		\end{enumerate}
	Ready to Running: Dispatcher!!!

	\item ps = In der Shell laufenden Prozesse
		\begin{enumerate}
			\item Virtuelles Verzeichnis /proc
			\item 
			\item Sehr leicht auf Systeminformation zugreifen
			\item 
				-A = Select all processes
				-l = Long format
				-p = Process ID
			\item PID, PPID (Parent Process), PRI = Priority
			\item bash
			\item bash has the same pid as the ppid of ps
			\item Wurzelprozess = Init/systemd ppid=0
			\item nach oben weitergegeben
		\end{enumerate}
	\item Nicht-präemptive FCFS (First Come First Serve) / SJF (Shortest Job First)
		Niedrige Process ID comes first
		\begin{enumerate}
			\item
			\item Verweilzeit (Ready + Running), Wartezeit ( = Ready)
			VWZ:
				FCFS = (2 + 4 + 11 + 12 + 13)/5 = 8,4
				SJF = (2+4+16+7+3)/5 = 6,4
			Wartenzeit:
				FCFS = (0+0+4+9+11)/5 = 4,8
				SJF = (0+0+9+4+1)/5 = 2.8
			\item SJF -> Langen Prozess entweder kommen sehr spät dran (Lange Prozesse haben eine sehr hohe Wartezeit, können verhungern)
		\end{enumerate}
\end{enumerate}