\begin{enumerate}[label={Aufgabe H\arabic*},start=10]
	\item 
		\begin{enumerate}
			\item Bei Multiprogramming wird der Prozessor zwischen mehreren Prozessen hin- und hergeschaltet. Jeder Prozess wird jedes Mal nur für einige 10 bis 100 Millisekunden ausgeführt, dann wird es unterbrochen. Zu jedem Zeitpunkt wird nur 1 Prozess ausgeführt. Da das CPU sehr schnell ist, scheint es aber, dass mehrere Prozesse gleichzeitig abgelaufen haben. 

			Das kann man als pseudo-parallele Ausführung von Prozessen nennen. 
			\item Man kann mehr Programmen in weniger Zeit ausführen. Das heißt auch, dass die Mittlerantwortzeit aller Prozesse reduziert sind.
			\item 
				\begin{enumerate}[label={(\roman*)}]
					\item Es gibt insgesamt 10 Möglichkeiten:
						\begin{multicols}{2}
							\begin{enumerate}[label={\texttt{[\arabic*]}}]
								\item \texttt{ABCDE}
								\item \texttt{ABDCE}
								\item \texttt{ABDEC}
								\item \texttt{ADBCE}
								\item \texttt{ADBEC}
								\item \texttt{ADEBC}
								\item \texttt{DABCE}
								\item \texttt{DABEC}
								\item \texttt{DAEBC}
								\item \texttt{DEABC}
							\end{enumerate}
						\end{multicols}
					\item Das bedeutet, dass die Anweisung \texttt{E} muss nach der Anweisung \texttt{B} ausgeführt werden. Das passiert leider in 3 von 10 obengennanten Möglichkeiten nicht: \texttt{[ADEBC]}, \texttt{[DAEBC]} und \texttt{[DEABC]}. In diesen Fallen kann es dann zum Fehler beim Datenlesen führen, z. B. Segmentation Fault. Anweisung E kann vielleicht auch einfach nicht durchgeführt werden können.

					Unter solchen Bedingungen muss es in dem Programm dann explizit gemacht, dass die Anweisung \texttt{B} nach der Anweisung \texttt{E} durchgeführt werden muss. 
				\end{enumerate}
		\end{enumerate}
	\item Sehen Sie bitte \texttt{u02-h11.txt}
\end{enumerate}