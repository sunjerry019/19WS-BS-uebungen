\newcommand{\SF}[1]{\textcolor{red}{#1}}

\begin{enumerate}[label={Aufgabe H\arabic*},start=60]
    \item \blanko
        \begin{center}
            \begin{tabular}{llllll}
                \toprule
                Zeit & Ref & $f_0, t$ & $f_1, t$ & $f_2, t$ & $\sum$ SF \\
                \midrule
                1  & 2 & \SF{2, 1}  & -          & -          & 1 \\
                2  & 1 &     2, 1   & \SF{1, 2}  & -          & 2 \\
                3  & 3 &     2, 1   &     1, 2   & \SF{3, 3}  & 3 \\
                4  & 4 & \SF{4, 4}  &     1, 2   &     3, 3   & 4 \\
                5  & 2 &     4, 4   & \SF{2, 5}  &     3, 3   & 5 \\
                6  & 2 &     4, 4   &     2, 6   &     3, 3   & 5 \\
                7  & 0 &     4, 4   &     2, 6   & \SF{0, 7}  & 6 \\
                8  & 1 & \SF{1, 8}  &     2, 6   &     0, 7   & 7 \\
                9  & 1 &     1, 9   &     2, 6   &     0, 7   & 7 \\
                10 & 3 &     1, 9   & \SF{3, 10} &     0, 7   & 8 \\
                11 & 1 &     1, 11  &     3, 10  &     0, 7   & 8 \\
                12 & 2 &     1, 11  &     3, 10  & \SF{2, 12} & 9 \\
                13 & 1 &     1, 13  &     3, 10  &     2, 12  & 9 \\
                14 & 4 &     1, 13  & \SF{4, 14} &     2, 12  & 10 \\
                15 & 4 &     1, 13  &     4, 15  &     2, 12  & 10 \\
                16 & 1 &     1, 16  &     4, 15  &     2, 12  & 10 \\
                \bottomrule
            \end{tabular}
        \end{center}
        \vspace{\baselineskip}
        
        Es gibt insgesamt 10 Seitenfehler.
    \item 
        \begin{enumerate}
            \item 
                \begin{enumerate}[label={(\roman*)}]
                    \item $w(5,6) = 5$ \hspace{1cm} $W(5,6) = \makeset{1, 6, 3, 7, 8}$ 
                    \item $w(7,4) = 3$ \hspace{1cm} $W(7,4) = \makeset{3, 7, 8}$
                    \item $w(8,3) = 2$ \hspace{1cm} $W(8,3) = \makeset{3, 8}$
                \end{enumerate}
            \item $h$ kann durch das Knie Kriterium gewählt werden. 

                Wir betrachten zunächst die Lifetime-Funktion $L(m)$, die die mittlere Zeit zwischen aufeinanderfolgenden Seitenfehlern in Abhängigkeit von der zugeordneten Rahmenanzahl $m$ an. Diese Lifetime-Funktion $L(m)$ steigt mit steigende $m$. Die Steigung bleibt nicht konstant und ändert sich entlang die $m$-Achse. Es gibt dann wenigstens ein "Knie", wobei die Funktion sich von einer stärken Steigung zu einer niedrigen Steigung ändert. Dabei kann eine Tangente von dem Nullpunkt auf der Graph der Funktion an einem Knie gezeichnet werden. Dieser Schnittpunkt ergibt dann eine Annäherung für das optimales Rahmenanzahl $h$.
        \end{enumerate}
    \item Sehen Sie bitte \texttt{u12-h62.txt}
\end{enumerate}