\begin{enumerate}[label={Aufgabe H\arabic*},start=45]
    \item 
        \begin{enumerate}
            \makeatletter
                \setlength{\leftmargins}{\@totalleftmargin}
            \makeatother

            \item Es gibt die folgende kritische Bereiche:
                \begin{itemize}
                    \item Türbereich betreten
                    \item Barhocker besitzen
                    \item Getränke von dem Barkeeper bestellen
                \end{itemize}
            \item \blanko
                \vspace{-1em}
                \begin{center}
                    \small
                    \begin{tabularx}{0.85\textwidth}{p{3cm}X}
                        \toprule
                        Bezeichner & Zweck \\
                        \midrule
                        \texttt{bestellwunsch} & Signalisiert dem Barkeeper einen Bestellwunsch\\
                        \texttt{getränk\_fertig} & Signalisiert einem Gast die Fertigstellung seines Getränks \\
                        \texttt{barkeeper\_frei} & Signalisiert einem Gast ob der Barkeeper derzeit frei ist \\
                        \texttt{tür\_frei} & Signalisiert einem Gast ob die Tür frei ist \\
                        \texttt{barhocker\_frei} & Zeigt, wie viele freie Barhocker es gibt \\
                        \bottomrule
                    \end{tabularx}
                \end{center}
                \vspace{\baselineskip}
            \item \blanko
                \vspace{-1em}
                \begin{center}
                    \small
                    \begin{tabularx}{0.85\textwidth}{p{4.5cm}X}
                        \toprule
                        Pseudocode & Bedeutung \\
                        \midrule
                        \texttt{init(bestellwunsch, 0);} & Initialisiert den Semaphor \texttt{bestellwunsch} mit dem Wert 0 \\
                        \texttt{init(getränk\_fertig, 0);} & Initialisiert den Semaphor \texttt{getränk\_fertig} mit dem Wert 0 \\
                        \texttt{init(barkeeper\_frei, 1);} & Initialisiert den Semaphor \texttt{barkeeper\_frei} mit dem Wert 1. Er ist am Anfang frei. Dieser Semaphor stellt es sicher, dass nur ein Gast mit dem Barkeeper interagieren kann.\\
                        \texttt{init(tür\_frei, 1);} & Initialisiert den Semaphor \texttt{tür\_frei} mit dem Wert 1. Es ist am Anfang frei. \\
                        \texttt{init(barhocker\_frei, 5);} & Initialisiert den Semaphor \texttt{barhocker\_frei} mit dem Wert 5. Es gibt am Anfang 5 freie Barhockern\\
                        \bottomrule
                    \end{tabularx}
                \end{center}
            \pagebreak[4]
            \item \blanko 
                \vspace{\baselineskip}

                \begin{minted}[linenos,firstnumber=last,autogobble,xleftmargin=-\leftmargins,frame=leftline,framesep=10pt]{java}
                Gast() {
                    wait(tür_frei);
                    <die Bar betreten>;
                    wait(barhocker_frei); // Warten im Türbereich bis Barhocker frei ist
                    <auf Hocker Platz nehmen>;
                    signal(tür_frei);

                    for(Getränke = 0; Getränke < 3; Getränke ++) {
                        wait(barkeeper_frei);
                        signal(bestellwunsch); // Bestellung aufgeben
                        wait(getränk_fertig);  // Zubereitung abwarten
                        <Getränk entgegennehmen>;
                    }

                    wait(tür_frei);
                    // Den Türbereich betreten
                    signal(barhocker_frei);
                    <die Bar verlassen>;
                    signal(tür_frei);
                }
                Barkeeper() {
                    while(true) {
                        // auf Bestellung warten
                        wait(bestellwunsch);
                        //Fertigstellung signalisieren
                        signal(getränk_fertig);
                        <Getränk an den Gast geben>;
                        signal(barkeeper_frei);
                    }
                }
                \end{minted}
            \vspace{\baselineskip}
        \end{enumerate}
    \item Sehen Sie bitte \texttt{u09-h46.txt}
\end{enumerate}
